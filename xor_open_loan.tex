%%%%%%%%%%%%%%%%%%%%%%%%%%%%%%%%%%%%%%%%%
% Journal Article
% LaTeX Template
% Version 1.3 (9/9/13)
%
% This template has been downloaded from:
% http://www.LaTeXTemplates.com
%
% Original author:
% Frits Wenneker (http://www.howtotex.com)
%
% License:
% CC BY-NC-SA 3.0 (http://creativecommons.org/licenses/by-nc-sa/3.0/)
%
%%%%%%%%%%%%%%%%%%%%%%%%%%%%%%%%%%%%%%%%%

%----------------------------------------------------------------------------------------
%	PACKAGES AND OTHER DOCUMENT CONFIGURATIONS
%----------------------------------------------------------------------------------------

\documentclass[twoside]{article}

\usepackage{graphicx}

\usepackage{booktabs,tabularx}

\usepackage{lipsum} % Package to generate dummy text throughout this template

\usepackage[sc]{mathpazo} % Use the Palatino font
\usepackage[T1]{fontenc} % Use 8-bit encoding that has 256 glyphs
\linespread{1.05} % Line spacing - Palatino needs more space between lines
\usepackage{microtype} % Slightly tweak font spacing for aesthetics

\usepackage[hmarginratio=1:1,top=32mm,columnsep=15pt]{geometry} % Document margins
\usepackage{multicol} % Used for the two-column layout of the document
\usepackage[hang, small,labelfont=bf,up,textfont=it,up]{caption} % Custom captions under/above floats in tables or figures
\usepackage{booktabs} % Horizontal rules in tables
\usepackage{float} % Required for tables and figures in the multi-column environment - they need to be placed in specific locations with the [H] (e.g. \begin{table}[H])
\usepackage{hyperref} % For hyperlinks in the PDF

\usepackage{lettrine} % The lettrine is the first enlarged letter at the beginning of the text
\usepackage{paralist} % Used for the compactitem environment which makes bullet points with less space between them

\usepackage{abstract} % Allows abstract customization
\renewcommand{\abstractnamefont}{\normalfont\bfseries} % Set the "Abstract" text to bold
\renewcommand{\abstracttextfont}{\normalfont\small\itshape} % Set the abstract itself to small italic text

\usepackage{titlesec} % Allows customization of titles
\renewcommand\thesection{\Roman{section}} % Roman numerals for the sections
\renewcommand\thesubsection{\Roman{subsection}} % Roman numerals for subsections
\titleformat{\section}[block]{\large\scshape\centering}{\thesection.}{1em}{} % Change the look of the section titles
\titleformat{\subsection}[block]{\large}{\thesubsection.}{1em}{} % Change the look of the section titles
\usepackage{amsmath}



%----------------------------------------------------------------------------------------
%	TITLE SECTION
%----------------------------------------------------------------------------------------

\title{\vspace{-15mm}\fontsize{24pt}{10pt}\selectfont\textbf{XOR Open Loan}} % Article title
\author{\large\textsc{Jeff An, Matthew Black, Tony Cai, Justin Kennedy, Vidur Sanandan}\\[2mm]} % Your name


\date{}
%----------------------------------------------------------------------------------------

\begin{document}

\maketitle % Insert title


\begin{abstract}
XOR Open Loan is a unique open source and completely decentralized system that makes loan contracts that are low cost for borrowers, high return for lenders, and contain end-to-end transparency and immutability on the blockchain. XOR Open Loan removes the middleman underwriter by assigning a score analog to borrowers created by a distributed consensus from other borrowers with a Trust Score, which is the XOR Open Loan analog to credit scores. With this, all administrative and personnel fees are removed, immediately making an XOR Open Loan orders of magnitude more cost effective than a regular loan contract signed by a trusted third party. In addition, because lenders are themselves incentivized to make a maximum return on investment, they are incentivized to make the fee-to-interest ratio on the loan as low as possible. This creates a situation where both parties get the best possible deal. This has huge implications for regular borrowers as well as borrowers from third-world countries, because creating access to transparent loans allows these people to leverage their personal wealth through loans and credit, enabling access to modern financial systems, thus massively and scaleably to escape from poverty.  Moreover, the XOR Open Loan system records all loan contracts on a public ledger on the blockchain. This means that all contracts are guaranteed to be transparent. The implications for a transparent, trustable loan contract system are huge; generally speaking, the concept of debt can be used to model almost any economic system. Thus, a transparent open source loan contract allows for the possibility of infinite numbers of derivative markets based on those debt obligations that can be created by anyone because of XOR Open Loan?s open source paradigm. In essence, anyone in the world has the ability to create their own market on anything. Possibilities range from markets based on sports bets to transparent CDOs where every tranch is available to view and verify. XOR Open Loan will bring leverage to everyone in the world, democratizing a power previously only held by large self-interested banks. The possibilities, which are up to the creativity and interest of the general public, are literally endless.
\end{abstract}
\section{Introduction}
When an XOR participant seeks a loan, he fills out a request specifying a set of parameters, including loan amount and maturity period which is then used in conjugation with the requests of other borrowers to form a market based on common interests. The resulting market in effect represents a common risk profile of borrowers using the aggregate of their trust scores and loan amount requests as a proxy. Lenders on the lender network submit a loan intent to the market they wish to invest in that includes the amount they wish to loan and the minimum interest rate they will accept. If there is demand in the market for the loan specifications the lender is offering, the intent is approved and an algorithm is utilized to match the lender’s loan with a borrower’s needs.
Additionally, each market has a  request period duration that reflects the rate at which new borrowing requests are 'settled' with lenders in the market that is voted on by the borrowers and lenders at the initiation of the market. Once a market is formed, lenders vote on a margin requirement as well as insurance rate. The margin requirement represents the down payment in percentage terms that borrowers in the market have to put down at the start of each loan transaction. The insurance rate reflects the extent to which lenders will be compensated in case of default by one or more of the borrowers and consequently the collateral each lender puts down at the start of the request duration period.

\section{Markets}
\subsection{Basic Markets}
A basic market is a lightweight, user-created place of exchange for transactions that conform to some set of hyperparameters. These hyperparameters define a relatively consistent risk profile that borrowers in the market conform to. Examples of hyperparameters include loan maturity period, maximum loan amount, trust score range, and request period duration, which controls the rate at which new borrowing requests are “settled”. Each basic market may also define an insurance rate, which determines what portion of the interest fee is placed into a market-wide insurance pool in each loan transaction. The insurance pool, as mentioned previously, provides a safeguard against borrower defaults in riskier markets.

As a simple example, a medium-risk, medium maturity market might have a month-long maturity period, a low insurance rate, and a request period of 5 minutes. A microloan market, on the other hand, might have a maximum loan amount of \$100, no insurance rate, and a request period of 10 seconds.

Since basic markets have virtually no overhead, they can be created freely by anyone on the platform. The intended use case for basic markets is for communities where borrowers have an extremely consistent, predictable risk profile or are connected in real life in some fashion. For example, a small suburban community might create a basic market to handle short-term loans between community members.
\subsection{Managed Markets}
Managed markets differ from basic markets in that the interest and insurance rate calculations are carried out by a teller function, which is a piece of arbitrary code that takes as input the parameters of a loan (maturity period, borrower trust score, etc.) and outputs an insurance rate and interest rate. A managed market is associated with a teller function at creation time, a process that requires a fee corresponding to the cost of deploying the function as a contract to the blockchain network. Since the function resides on the blockchain, the code is transparent. Currently, a managed market cannot change its associated teller function once created.

Users are incentivized to create markets with “high accuracy teller” functions: functions which consistently output insurance and interest rates that justly reflect the underlying risk of the loan. This is accomplished by awarding a fixed percentage of each transaction carried out on a managed market to the market creator. Since users will only be willing to pay this transaction fee if the teller function has been historically accurate, it is expected that the managed markets that eventually become popular on the XOR network will serve as industrial-grade, high-throughput mediums of exchange between lenders and borrowers who have highly variable or even unknown risk profiles.

Machine learning models such as LSTMs will likely be well-suited for teller functions, and are expected to dominate managed markets. As XOR matures, training data will also become more readily available, creating a positive feedback loop in which teller functions become better and better at predicting defaults. One can easily imagine many potential connections here to academia as well.

\section{Loan Procedure}
\subsection{Rounds}
All markets operate in rounds, which are composed of a request period followed by settlement. During the request period, borrowers are able to request funds with a certain maturity date and maximum interest rate they are willing to pay. Lenders are also able to file loan intents during this period, which contain a single vote for the risk coefficient used to calculate interest rates in the market (described in more detail below). At the end of each round, loan requests and intents are reconciled by matching borrowers with the best possible interest rate. In the case of a tie, the lender who filed the loan intent first is matched with the borrower in question, incentivizing early entry into markets. As with other decentralized currencies, multiple lenders can fulfill a single loan request, increasing market liquidity.

\subsection{Loan Intents}
To supply funds to a certain market, an investor creates a loan intention (“loan intent”) with the specific amount of funds to be supplied, and the risk coefficient that the investor believes characterizes the current risk-reward profile in the market. The risk coefficient, as discussed earlier, directly determines interest rate and insurance pool calculations for each loan in the market. The risk coefficient specified by the investor amounts to a single vote; the actual coefficient that will be used to calculate the settlement interest rate for each loan is a weighted average of all such votes received within a round’s request period.

Technically, a loan intent is simply a smart contract indicating that a predetermined amount of funds will be transferred from the investor’s account into a particular market at the next settlement time. If the investor is not selected to participate in the round (i.e. since the supply of credit exceeds the demand for credit), the funds are simply returned to the investor. To remove unnecessary manual labor from the process of supplying funds, the XOR platform could provide a simple utility to reinvest unused funds by recreating rejected loan intents until one is accepted. This process closely models a “market order” in stock market terminology.

\subsection{Pricing}
In order to hedge against defaults, lenders have to post a collateral fee to the insurance pool. This fee is calculated as a percentage of interest fees earned in every loan. This percentage, termed $\phi$, is calculated every round as a weighted average of the desired insurance rates $\phi\prime$ included in each of the loan intents submitted during a round. Thus, the final price for a given loan takes into account the interest rate specified in the loan intent and the aggregate insurance rate of the market:

\begin{equation}
V_{price} = a\times V_{loan} + b\times \phi_{market}
\end{equation}
where a is the interest rate offered by the lender (which must be lower than the maximum interest rate indicated by the borrower), and b is voted upon by all lenders participating in the market during a round. 

\section{Trust Protocol}
A robust trust protocol is required in order to allow lenders to verify borrowers without a third party underwriter. The two situations that are possible are that a borrower has a Trust Score on the system, and the borrower does not have a score. In the event that the borrower does not have a score, our process of Social Verifiability will provide a Trust Score automatically. 
\subsection{Trust Score}
The Trust Score is the XOR Open Loan analog to a credit score. A high Trust Score indicates that a borrower is highly trustable on the network. This means that the borrower's opinion on vetting low/no credit customers is highly valuable. As a reward for having a high Trust Score, the borrower will have access to more cash and better interest rates from lenders. A high Trust Score indicates a low risk profile for that borrower. 

Borrowers can increase their Trust Score $T_{n}$ by not defaulting on loans and by correctly verifying low/no credit score customers that see through their contracts. Trust Scores will decrease slightly if a low/no credit verification fails and will decrease drastically in the event of a default. 
\begin{equation}
T_{n,new} = T_{n} + \sum\limits_{j=1}^f  \frac{1}{T_{j,succes}} + \sum\limits_{k=1}^g  \frac{d}{T_{j,failure}}
\end{equation}
where $T_{n,new}$ is the new score at the end of a round of verifying, $T_{n}$ is the original score, $j$ is the index of a borrower being vetted by the verifier successfully, $f$ is the total number of borrowers verified within a round successfully, and $T_{j,success}$ is the score of one of the borrowers being verified successfully. $k$ is the index of a borrower defaulting a loan, g is the number of defaulted loans within a round, and $T_{j,failure}$ is the score of a defaulted borrower. $d$ is a positive real number weight constant that ensures that verifying a default penalizes a verifier more than a successful verification. This is necessary because without such a mechanic, verifying good borrowers would not be incentivized at all. The reciprocal of the $T$ allows for lower scores to have a higher impact. For example, in the event of a low-risk borrower defaulting, the penalty should not be high because the verifier was justified in thinking that the low-risk borrower would not default.
 
\subsection{Social Verifiability}
In the event of a customer without a Trust Score requesting a loan, there are two primary options. The first is that the customer posts a loan request, and the contract is treated as extremely high risk. In this situation, the borrower would receive a low amount of cash and high interest. On the lender side, the lender will post a high collateral fee to the insurance pool, but will also gain high return on investment from the high interest. 

Another option for a customer without a Trust Score is to receive initial verification from borrowers who have Trust Scores. We call this scheme Social Verifiability because a borrower without a Trust Score can leverage his social circle to receive verification. In this scheme, having a high number of verifiers as well as verifiers with high Trust Scores will improve the customer's initial Trust Score. We can model this relationship with an equation that follows the form: 
\begin{equation}
   \sum\limits_{i=1}^R log(r_{i}) - \sum\limits_{i=1}^D log(d_{i})  + sum\limits_{i=1}^V log(v_{i}) - \sum\limits_{i=1}^VS log(vs_{i})
\end{equation}
where $T_{n,init}$ is the new borrower's initial Trust Score, $q$ is the index of every verifier, $h$ is the total number of verifiers, $T_{q}$ is the Trust Score for a given verifier, $r$ is a weight that decreases $T_{init}$, and $u$ is a weight that decreases the effect of the number of verifiers. Without $r$ and $u$, the new borrower's Trust Score would essentially be just an average of the verifiers' scores plus a number parameter. Verifier scores are affected as per Equation (5). 

\section{Loan Completion and Defaults}
XOR Open Loan is agnostic to time parameters of the loan, meaning that contract types ranging from short-term completion timelines to perpetual swaps can be supported flexibly. Upon a contract being completed, the borrower will return all that is owed, initial amount plus interest, directly to the lender. 

In the event of a default, several things happen to both sides. The borrower's Trust Score drops significantly. For the seller, the full amount without interest is paid back from the insurance pool. In order to protect lenders against an event in which one lender's large position defaults and thus greatly reduces the insurance pool, insurance payments will be paid over time. 

\section{Portfolios}
Portfolios in XOR, in a manner analogous to mutual funds in the traditional financial system, allow investors to organize their investments. Just as markets aggregate borrower requests, portfolios aggregate a user’s loan intents. This additional layer of abstraction allows XOR to natively support investment diversification; since each market represents a relatively consistent risk profile, holding a portfolio with many different markets constitutes diversification.

Just like a mutual fund, it is possible for many lenders to invest in a single portfolio, which simply means that each investor creates loan intents in the same markets and supplies funds to each market in the same proportions. This allows “layman” investors who are not capable of analyzing the individual risk-reward profiles of individual markets, as well as investors who do not have the time to construct their own portfolio, to choose a suitable portfolio to invest in out of the current well-performing portfolios. As with everything in XOR, the performance of a portfolio is completely transparent; statistics on past returns and losses can be computed by any user since historical information on the underlying markets is freely available.

Finally, as portfolios represent a group of investments in different markets, it follows naturally that portfolios can be combined with other portfolios. This allows portfolios to be iterated upon seamlessly.


\section{Derivative Markets Examples}
A debt contract is the most basic asset class. Everything can be modeled as a series of debt obligations. This is perhaps the most interesting use case for XOR Open Loan. An opensourced, trustable, incentivized fair system for a debt contract allows for derivative assets to be modeled based on the debt contract as the underlying asset. We provide a few examples of common asset classes that can be modeled with XOR Open Loan. 
\subsection{XOR Credit Default Swaps}
Section II.II describes a contract risk pricing model. When a lender assigns a valuation in XOR Open Loan, he is essentially buying a Credit Default Swap, but with the insurance provider being the collective pool of lenders. This provides a huge opportunity for accurately priced, essentially no-fee, credit default swap contracts. These insurance purchases can then be chained as underlying assets. 
\subsection{XOR Collateral Debt Obligations}
An XOR Open Loan Collateral Debt Obligation would also be an incredibly impactful event in finance. The tranches of the CDO can be comprised of loans of different risk tiers. Senior tranches could receive the first payments as the tranches descend to high risk borrowing customers. 
\subsection{XOR Synthetic CDOs}
More interesting is the potential for synthetic CDOs. A synthetic CDO is essentially a gamble on the performance of another CDO. These gambles can chain with increasing odds, leading to a massive amount of leverage in the market that otherwise didn't exist. Synthetic CDOs when used correctly are healthy for a market because they introduce liquidity and leverage. However, many synthetic CDOs have opaque underlying assets. 

An XOR Open Loan synthetic CDO would be truly revolutionary because of the blockchain architecture. All tranches of the CDO would be fully viewable due to the public ledger of the blockchain recording every step of the creation of the CDO. Thus, XOR can enable a healthy and relatively safe form of CDO leverage and liquidity. 
\subsection{Creation of Random Markets}
An especially unique facet of XOR Open Loan is that it enables anyone to create any type of market. As explained previously, a debt obligation contract can be used to generally model a myriad of economic systems. Because of this, the XOR Open Loan system allows for the creation of markets by anyone within the framework of fairness, accessibility, and trustworthiness built-in. For example, one could create a sports betting market, or a market betting on the weather, or even some sort of crazy nth order synthetic CDO market where the underlying assets are cryptocurrencies pegged to weather patterns. As long as there are lenders and loaners, any market can exist through XOR Open Loan. 

\section{Transparency and Accessibility}
XOR Open Loan is licensed under the open-source MIT License. This is extremely important to the full functionality of the product. The goal of XOR Open Loan is to allow for transparent and widespread trading and market making. An important facet of this is the blockchain architecture. Because XOR Open Loan contracts are stored on the blockchain, they can all be accessed in their entirety. 

A big problem with the subprime mortgage crisis was that the tranches of the CDOs used were relatively inaccessible to most people. The opaqueness of the tranches prevented most investors from seeing the obvious pitfalls of subprime mortgage loans. However, with XOR Open Loan, all of the tranches will be entirely accessible. This allows for a far healthier and safer implementation of CDOs, and enables even synthetic CDOs to be a healthy financial instrument. 

\section{Democracy and Social Impact}
We believe that market making, trading, and financial asset synthesis should be available to anyone. In addition, we believe that these things greatly enable credit for those who could need it by creating leverage and allowing for loans. In many third world countries, a better life is bottlenecked because members of these countries do not have access to modern global economics. However, XOR Open Loan's open source public architecture allows for anyone with internet to create their own wealth. We think that this is a massively impactful technology that will help a lot of people. 

\section{Problems and Future Additions}
In the timeframe of the hackathon, we were unable to complete everything necessary for the XOR Open Loan system to fully work. A big example is proof-of-identity. If a borrower does not have to prove identity, then that borrower can make $x$ copies of himself and increase his leverage x-fold without actually deserving it. A proof-of-identity system is absolutely required to avoid this. A proof-of-identity implementation is well beyond the scope of one hackathon, but is still very much possible. 

Another addition we would make is integration with 0x. A decentralized exchange is the perfect place to trade decentralized creative assets. Enabling seamless 0x integration would provide a strong incentive to use XOR Open Loan. 
%----------------------------------------------------------------------------------------
\end{document}
